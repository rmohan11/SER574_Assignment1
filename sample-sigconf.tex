%
% The first command in your LaTeX source must be the \documentclass command.
\documentclass[sigconf]{acmart}
%
% defining the \BibTeX command - from Oren Patashnik's original BibTeX documentation.
\def\BibTeX{{\rm B\kern-.05em{\sc i\kern-.025em b}\kern-.08emT\kern-.1667em\lower.7ex\hbox{E}\kern-.125emX}}
    
% Rights management information. 
% This information is sent to you when you complete the rights form.
% These commands have SAMPLE values in them; it is your responsibility as an author to replace
% the commands and values with those provided to you when you complete the rights form.
%



%%%%%%%%%%%%%%5
\usepackage{lipsum}


\begin{document}

%
% The "title" command has an optional parameter, allowing the author to define a "short title" to be used in page headers.
\title{Collaborating Between Agile Teams}

%
% The "author" command and its associated commands are used to define the authors and their affiliations.
% Of note is the shared affiliation of the first two authors, and the "authornote" and "authornotemark" commands
% used to denote shared contribution to the research.
\author{Rakesh Mohan}
\author{Venkata Akhil Madaraju}

%
% The abstract is a short summary of the work to be presented in the article.
\begin{abstract}
A clear and well-documented \LaTeX\ document is presented as an article formatted for publication by ACM in 
a conference proceedings or journal publication. Based on the ``acmart'' document class, this article presents
and explains many of the common variations, as well as many of the formatting elements
an author may use in the preparation of the documentation of their work.
\end{abstract}


%
% Keywords. The author(s) should pick words that accurately describe the work being
% presented. Separate the keywords with commas.
\keywords{datasets, neural  networks, gaze detection, text tagging}

%
% A "teaser" image appears between the author and affiliation information and the body 
% of the document, and typically spans the page. 


%
% This command processes the author and affiliation and title information and builds
% the first part of the formatted document.
\maketitle


\section{Introduction}
The rise in the competition of in every industry means there is an increase in the uncertainty and volatility in every field which has resulted in an increase in the complexity of the software's. The increase in the complexity means the requirements of software is getting harder and harder to describe at the beginning of the project leading to a lot of ambiguity and incompleteness. That is why it so necessary to design for change rather execute the plan which also happens to be one of the manifesto of the agile methodology. Any new development these days are being managed by forming a small team with a team of developers having expertise in the required subject area. As more and more companies prefer agile methodology for project development it has led to situations when different agile teams have to work with each other to obtain a final end product. While this may sound easy for different teams to work together technically as they are implementing the same methodology there are so many hindrances that impedes the development process.

In the blog \cite{Alia01} Alia Crocker, Rob Cross and Heidi K. Gardner discuss the issues faced by the organization Connected Commons which came upon a new innovative ground-breaking audio-visual technology which had the potential to open up an entirely new markets for the organization. Their CEO considered it to be "pivot point" in organizations growth and created a cross functional group of teams to develop the application. But different teams assigned to this task struggled to develop the product as often they had problem in understanding the expertise or values of different function of other teams. At the same time fighting tooth and nail advocating for their own solution as aggressively as possible. Each of the teams were surprised by the requirement of the external stakeholders thus showing how much the collaboration between the agile teams could be an hinderance to the progress in the development and not matching the expectations of the progress. This issue is not specific or unique to this organization alone it is quite the same situation whenever there is a necessity for different agile teams to work together.  This doesn't mean that there is no way for the agile teams to collaborate there definitely ways for teams to collaborate better if they could follow certain guidlines of effective collaboration.

\section{Effective Collaboration}

According to the GSA guide \cite{Gsa01} there are three important guidelines that needs to be followed for the effective collaboration of the agile teams. They are,

\begin{itemize}
  \item Clear objectives and separation of work
  \item No departmental silos
  \item Collaborative architecture and design guidelines
\end{itemize}
\subsection{Clear objectives and separation of work}

It is necessary to specify the Objectives of the team and Separation of work clearly so that eaach team is clear about the expectations on them. When there is a clarity in the objectives of each team there is independence between the team which reduces the possibility of reduntant work. No two team will do the same work due to clear separation of work this can be achieved by creating and maintaining separate product backlog for each of the team rather than one common one and it is very important to have the big picture in the mind. 

\subsection{No departmental silos}
For effective collaboration form agile teams whose members are cross functional so that the members will be able to go beyond the boundaries of departments. Thus helps in creating a culture that eliminates deparmental silos that could barricades the over all understanding and thus hindering the progress in achieving the final product.

\subsection {Collaborative architecture and design guidelines}
The final and perhaps the most important guidelines for most effective collaboration is the collaborative architecture and design guidelines. The organization should provide an environment that is more supportive of collaboration as agile teams are self organizing teams and the best design are obtained such an self organized team. Providing the supportive environment will help the agile teams to collaborate better, self organize and come up with an architecture that all the stake holders can agree upon thus helping with the smooth progress with the development.

\section{Agile Architecture}
Agile Architecture is that which supports evolution of design and architecture of the system when implementing the new system capabilities.With Agile, the software should always run and is delivered in increments.The incremental delivery is done by balancing between the two details which are:
\begin{itemize}
  \item Emergent Design - Agile comes with a completely evolutionary and incremental implementation approach. This help the developers respond to the requirements based on the user priorities which allows the design to emerge as the system is built and deployed. 
  \item Intentional architecture - Set of planned architectural initiatives that helps building the solution and also provides guidance for inter-team design and implementation synchronization.
\end{itemize}

\section{Design Emerges. Architecture is a Collaboration}
The Agile Manifesto says that "The best architectures, requirements, and designs emerge from self-organizing teams." Teams working on the software evolve the design accordingly with the current set of requirements. The best architectures, requirements, and the design emerge from self-organizing teams. The agile teams working with other teams will help build an architecture that fits team needs and requirements of enterprise. The team reflects on how to become more effective as the design emerges and then it adjusts and tunes its behavior accordingly.

Agility and the design in Agile is possible through greater discipline of everyone working in the process. It is seen that the software architects and agile development teams have a mixed history of working together, this is unfortunate but rectifiable. 

\section{Architecture work that supports Agile Development}
There are certain architecture practices that help the architecture support the Agile process. These practices are derived from the Agile Manifesto. It also includes the teams working together with others and collaboration such as:

\subsection{Capture clear architecture principles}
Agile teams will need each team member and the team to be able to make good design decisions. The teams also need to understand why the architectural structures exist and the architecture's most important characteristics. This helps when the agile team works together with other teams when extending or adapting for a change in the architecture.

\subsection{Define components clearly}
As the systems evolve and the new components are introduced, the existing ones are seen to change and the component interactions are altered. The teams need to define clear set of responsibilities and a set of required dependencies for the architecture and the deign system to evolve coherently.

\subsection{Collaboration over Contracts}
The agile design should allow the teams to collaborate rather than maintain formal boundaries and agreements. Collaboration is the key to effective architecture work and is supported by the Agile manifesto.

\subsection{Focus design work on stakeholder concerns}
One of the aspects of collaboration is that the architectural design work aligns with the needs of the stakeholder and the teams working on it with others, this also works on systems most important aspects. This can be achieved by the inter working teams by using common design patterns, how the system will be deployed and other critical qualities such as security and availability of the system.

\section{Conclusion}
It is not so easy for different agile teams to work together as they are implementing the same process and design plays a key role for this collaboration. Although agile collaboration requires continuous re-assesment of complex problems, it is possible for firms to combine and recombine essential enterprise from across points in the network to address the design issues.  

%
% The next two lines define the bibliography style to be used, and the bibliography file.
\bibliographystyle{ACM-Reference-Format}%alpha
\bibliography{reference}

\end{document}
