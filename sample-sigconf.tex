%
% The first command in your LaTeX source must be the \documentclass command.
\documentclass[sigconf]{acmart}
%
% defining the \BibTeX command - from Oren Patashnik's original BibTeX documentation.
\def\BibTeX{{\rm B\kern-.05em{\sc i\kern-.025em b}\kern-.08emT\kern-.1667em\lower.7ex\hbox{E}\kern-.125emX}}
    
% Rights management information. 
% This information is sent to you when you complete the rights form.
% These commands have SAMPLE values in them; it is your responsibility as an author to replace
% the commands and values with those provided to you when you complete the rights form.
%



%%%%%%%%%%%%%%5
\usepackage{lipsum}


\begin{document}

%
% The "title" command has an optional parameter, allowing the author to define a "short title" to be used in page headers.
\title{Collaborating Between Agile Teams}

%
% The "author" command and its associated commands are used to define the authors and their affiliations.
% Of note is the shared affiliation of the first two authors, and the "authornote" and "authornotemark" commands
% used to denote shared contribution to the research.
\author{Rakesh Mohan}
\author{Venkata Akhil Madaraju}

%
% The abstract is a short summary of the work to be presented in the article.
\begin{abstract}
The Agile development process is one of the most commonly used development process in today's ever-changing world. As it is almost impossible to come up with the full list of requirements which is complete and unambiguous before beginning with the development process as a result of which the traditional process of development has almost become obsolete in most of the commercial areas of development with exception of safety critical system. In such a trend it is unavoidable having different agile teams working with each other, but while collaborating between the agile teams their performs drops due to the inefficient collaboration between them thus hindering the progress of development and in a way negating the benefits of Agile development process. Since there is no sight of change in this trend in near future it is very essential for us to come up with better ways of collaboration between the agile teams so that they perform to their fullest potential leading to an unhindered progress. This paper suggests certain set guidelines that can be followed by the development team while making their design and architectural decisions and during the development to effectively collaborate.
\end{abstract}


%
% Keywords. The author(s) should pick words that accurately describe the work being
% presented. Separate the keywords with commas.
\keywords{agile manifesto, design, architecture, collaboration}

%
% A "teaser" image appears between the author and affiliation information and the body 
% of the document, and typically spans the page. 


%
% This command processes the author and affiliation and title information and builds
% the first part of the formatted document.
\maketitle


\section{Introduction}
The rise in the competition in every industry means there is an increase in the uncertainty and volatility in every field, which has resulted in an increase in the complexity of the software's. The increase in the complexity means the requirements of software is getting harder and harder to describe at one go in the beginning of the project leading to a lot of ambiguity and incompleteness. That is why it so necessary to design for change rather execute the plan which also happens to be one of the manifesto of the agile methodology. Any new development these days are being managed by forming a small team with a team of developers having expertise in the required subject area. As more and more companies prefer agile methodology as it is more convenient and efficient in most scenarios for project development it has led to situations where different agile teams have to work with each other with each working on different aspects of a product but at the same time working towards the same goal. While this may sound simple and the different teams should have no problem to work with each other as technically as they are following the same methodology there are so many instances where it has been found to be a hindrance that impedes the development process.

In the blog \cite{Alia01} Alia Crocker, Rob Cross and Heidi K. Gardner discuss about the serious issues faced by the organization "Connected Commons" while trying to develop product based on a new innovative idea which they believed to be a ground-breaking audio-visual technology which had the potential to open up an entirely new market. They created a cross functional group of teams to develop the application. But different teams assigned to this task struggled to develop the product as often they had problem in understanding the expertise or values on different function of other teams. At the same time, they were fighting tooth and nail advocating for their own solution most of the time as aggressively as possible. Each team were surprised by the requirement and expectations of the external stakeholders. Thus, showing how collaborating between the agile teams is complicated and could be an hindrance to the progress of the development if not implemented effectively. This issue is not specific or unique to this specific organization it is the same situation whenever there is a collaboration different agile teams. The necessity to collaborate between the agile teams is not going to stop given the popularity of the process so it is necessary find better ways of collaboration. Below are given certain set of guidelines for agile teams to follow while collaborating for the effective collaboration.

\section{Effective Collaboration}

According to the GSA guide \cite{Gsa01} there are three set of important rules/guidelines that needs to be followed for the effective collaboration between the agile teams. They are,

\begin{itemize}
  \item Clear objectives and separation of work
  \item No departmental silos
  \item Collaborative architecture and design guidelines
\end{itemize}
\subsection{Clear objectives and separation of work}

It is very important to clearly specify the Objectives of the team and differentiate the tasks that needs to be worked on by different teams in other words Separation of work. This helps to clear the expectations of each team and they can plan accordingly on how to implement those tasks. When there is a clarity in the objectives of each team there is independence between the teams and it also reduces the possibility of redundant work of having multiple teams perform the same work. Maintaining a separate product backlog for each team rather than having a one common one also helps with the separation of work as no two team will do the same work due to clear segregation. 

\subsection{No departmental silos}
The departmental silos are a huge hindrance in the overall understanding of the product. When there are domain specific team there is not much communication between departments. That is why for effective collaboration it is necessary to form agile teams whose members are cross functional so that the members will be able to go beyond the boundaries of departments. Thus, helps in creating a culture that eliminates departmental silos that could barricades the overall understanding and thus hindering the progress in achieving the final product.

\subsection {Collaborative architecture and design guidelines}
The final and perhaps the most important instruction for most effective collaboration is the teams discussing and coming up with a collaborative architecture and design. The organization needs to provide an environment that is more supportive of the collaborative effort of the agile teams. The agile teams are self-organizing teams and the best design in an agile process is obtained through such a self-organized team. Providing the supportive environment will help the agile teams to collaborate better, self-organize and come up with an architecture with which all the stake holders can agree upon thus helping with the smooth progress with the development.

\section{Agile Architecture}
Agile Architecture is something that supports the evolution of the design and architecture of the system while implementing a new system capability. As it is commonly known with Agile methodology, the emphasis is more on the running software where the product is delivered in increments. The incremental delivery is done by balancing between the two details which are:

\begin{itemize}
  \item Emergent Design - Agile comes with a completely evolutionary and incremental implementation approach. This help the developers respond to the requirements based on the user priorities which allows the design to emerge as the system is built and deployed. 

  \item Intentional architecture - Set of planned architectural initiatives that helps building the solution and also provides guidance for inter-team design and implementation synchronization.
\end{itemize}

\section{Design Emerges. Architecture is a Collaboration}
As already discussed the Agile Manifesto states that "The best architectures, requirements, and designs emerge from self-organizing teams." Teams working on the software evolve the design accordingly with the current set of requirements. The best architectures, requirements, and the design emerge from self-organizing teams. The agile teams working with other teams will help build an architecture that fits team needs and requirements of enterprise. The team reflects on how to become more effective as the design emerges and then it adjusts and tunes its behavior accordingly.

Agility and the design in Agile is possible through greater discipline of everyone working in the process. It is seen that the software architects and agile development teams have a mixed history of working together, this is unfortunate but rectifiable.

\section{Architecture work that supports Agile Development}
There are certain architecture practices that help the architecture support the Agile process. These practices are derived from the Agile Manifesto. It also includes practices for the teams working together with others in collaboration such as:

\subsection{Capture clear architecture principles}
Agile teams will need each team member and the team to be able to make good design decisions. The teams also need to understand why the architectural structures exist and the architecture's most important characteristics. This helps when the agile team works together with other teams when extending or adapting for a change in the architecture.

\subsection{Define components clearly}
As the systems evolve and the new components are introduced, the existing ones are seen to change, and the component interactions are altered. The teams need to define clear set of responsibilities and a set of required dependencies for the architecture and the design system to evolve coherently.

\subsection{Collaboration over Contracts}
The agile design should allow the teams to collaborate rather than maintain formal boundaries and agreements. Collaboration is the key to effective architecture work and is supported by the Agile manifesto.

\subsection{Focus design work on stakeholder concerns}
One of the aspects of collaboration is that the architectural design work aligns with the needs of the stakeholder and the teams working on it with others, this also works on systems most important aspects. This can be achieved by the inter working teams by using common design patterns, how the system will be deployed and other critical qualities such as security and availability of the system.

\section{Conclusion}
In future there are going to be more situations where multiple agile teams will have to work together cohesively towards a common goal to achieve a product. Therefore, it is very important for the teams to learn to collaborate effectively which can be achieved by clearly specifying the objectives for the team and segregating their work. Having a cross functional team in which the developers actively communicate between different teams thus actively avoiding any grey area and finally active collaboration between teams while deciding on the architecture and design so that all the stakeholders can give their input and design a comprehensive architecture with all the requirements and limitations in mind which can adopt future changes with ease. Thus, allowing for the smooth progress of the development without any sudden surprises like having a change in requirement that causes a disruptive changes to the product that has been built up to that point.

%
% The next two lines define the bibliography style to be used, and the bibliography file.
\bibliographystyle{ACM-Reference-Format}%alpha
\bibliography{reference}

\end{document}
