%
% The first command in your LaTeX source must be the \documentclass command.
\documentclass[sigconf]{acmart}
%
% defining the \BibTeX command - from Oren Patashnik's original BibTeX documentation.
\def\BibTeX{{\rm B\kern-.05em{\sc i\kern-.025em b}\kern-.08emT\kern-.1667em\lower.7ex\hbox{E}\kern-.125emX}}
    
% Rights management information. 
% This information is sent to you when you complete the rights form.
% These commands have SAMPLE values in them; it is your responsibility as an author to replace
% the commands and values with those provided to you when you complete the rights form.
%



%%%%%%%%%%%%%%5
\usepackage{lipsum}


\begin{document}

%
% The "title" command has an optional parameter, allowing the author to define a "short title" to be used in page headers.
\title{Collaborating Between Agile Teams}

%
% The "author" command and its associated commands are used to define the authors and their affiliations.
% Of note is the shared affiliation of the first two authors, and the "authornote" and "authornotemark" commands
% used to denote shared contribution to the research.
\author{Rakesh Mohan}
\author{Venkata Akhil Madaraju}

%
% The abstract is a short summary of the work to be presented in the article.
\begin{abstract}
The Agile development process is one of the most commonly used development process in the current market. This is due to the fact that it is almost impossible to come up with the complete list of requirements which is unambiguous before beginning with the development process. As a result of this almost all the traditional development processes have become obsolete in most commercial areas of development with an exception of safety critical system. In such a trend it is unavoidable having different agile teams working with each other, but collaboration between agile team always causes a dip in their performance. Therefore, hindering the progress of development and in a way negating the benefits of Agile development process. As there is no sight of change in this trend in near future it is essential to come up with better ways of collaboration between agile teams. This paper suggests a certain set of guidelines that can be followed by the development teams while making their design and architectural decisions and during the development process to collaborate effectively.
\end{abstract}


%
% Keywords. The author(s) should pick words that accurately describe the work being
% presented. Separate the keywords with commas.
\keywords{agile manifesto, design, architecture, collaboration}

%
% A "teaser" image appears between the author and affiliation information and the body 
% of the document, and typically spans the page. 


%
% This command processes the author and affiliation and title information and builds
% the first part of the formatted document.
\maketitle


\section{Introduction}
In today's competitive markets every industry is volatile and uncertain which meant an increase in the complexity of software. The increase in the complexity meant a lot of changes to the requirements well in to the development process. That is the reason it is necessary to design for change rather than execute a plan. This also happens to be one of the manifesto of the agile development process. 

Typically, these days to develop a new application a small team of developer with required area of expertise is put together. Most companies prefer this method of development as it is comparatively more effective and convenient. If developing a more complex system multiple agile teams work together towards achieving the same product. In which each of these teams concentrate on different aspects of the product development. Though it can be assumed that the agile teams collaborating will be as effective as them working alone. There have been many instances in the past where the progress has been hindered due to the poor collaboration of agile teams.

In their blog \cite{Alia01} Alia Crocker, Rob Cross and Heidi K. Gardner discuss about the serious issues faced by the organization "Connected Commons". When they were in the process of developing a new product based on an innovative idea which they believed to be a ground-breaking audio-visual technology. The company's CEO believed this had the potential to open up an entirely new market to them. They created different cross functional teams concentrating on different department to develop this application. But they had a hard time collaborating as they didn't understand the expectation from other teams and their restriction. 

At the same time, they were fighting hard advocating for their own expectation and restriction as aggressively as possible. Thus, proving collaborating between agile teams is very complicated and could hinder the progress of the development if implemented ineffectively. This issue is not unique to this specific organization it is the same situation whenever there are agile teams collaborating. The necessity to collaborate between the agile teams is not going away given the popularity of the process. Therefore, it is necessary to find better ways of collaborating between them. Below given are certain set of guidelines for effective collaboration of agile teams.


\section{Effective Collaboration}

According to the GSA guide \cite{Gsa01} there are three set of important rules/guidelines that needs to be followed for the effective collaboration between the agile teams. They are,

\begin{itemize}
  \item Clear objectives and separation of work
  \item No departmental silos
  \item Collaborative architecture and design guidelines
\end{itemize}
\subsection{Clear objectives and separation of work}

It is very important to clearly specify the objectives of different teams, differentiating their tasks from others, in other words separation of work. This sets a clear expectation for each team and they can plan accordingly to implement their tasks. When there is a clarity in the objectives there is no dependence between teams. Thus, reducing the possibility of redundancy in work, having multiple teams working on the same task. Maintaining a separate product backlog for each team rather than a common one helps in this aspect of separation of work as there is no confusion on who works on what, there is a clear segregation.

\subsection{No departmental silos}
The departmental silos are a huge hindrance in the overall understanding of the product. When there are domain specific teams there is not much of a communication between departments. That is why for effective collaboration it is necessary to form cross functional agile teams so that the members will be able to go beyond the boundaries of departments. Thus, helps in creating a culture that eliminates departmental silos that could hinder with the progress of development.

\subsection {Collaborative architecture and design guidelines}
The final and perhaps the most important suggestion for effective collaboration is team collaborating while making the architectural and design decisions. The organization needs to provide an environment that is supportive of agile teams collaborating. The best result is obtained from a self-organizing agile team. Thus, providing a supportive environment will help the agile teams to organize and collaborate better. As result of this the teams can come up with an architecture with which all the stake holders can agree upon thus helping with the smooth progress of development.

\section{Agile Architecture}
Agile Architecture is something that supports the evolution of the design and architecture of the system while implementing a new system capability \cite{Saf01}. As it is commonly known with Agile methodology, the emphasis is more on the running software where the product is delivered in increments. The incremental delivery is done by balancing between the two details which are:

\begin{itemize}
  \item Emergent Design - Agile comes with a completely evolutionary and incremental implementation approach. This help the developers respond to the requirements based on the user priorities which allows the design to emerge as the system is built and deployed. 

  \item Intentional architecture - Set of planned architectural initiatives that helps building the solution and also provides guidance for inter-team design and implementation synchronization.
\end{itemize}

\section{Design Emerges. Architecture is a Collaboration}
As already discussed the Agile Manifesto states that "The best architectures, requirements, and designs emerge from self-organizing teams." Teams working on the software evolve the design accordingly with the current set of requirements. The best architectures, requirements, and the design emerge from self-organizing teams. The agile teams working with other teams will help build an architecture that fits team needs and requirements of enterprise. The team reflects on how to become more effective as the design emerges and then it adjusts and tunes its behavior accordingly.

Agility and the design in Agile is possible through greater discipline of everyone working in the process. It is seen that the software architects and agile development teams have a mixed history of working together, this is unfortunate but rectifiable.

\section{Architecture work that supports Agile Development}
There are certain architecture practices that help the architecture support the Agile process. These practices\cite{Eoin1} are derived from the Agile Manifesto\cite{Am01}. It also includes practices for the teams working together with others in collaboration such as:

\subsection{Capture clear architecture principles}
Agile teams will need each team member and the team to be able to make good design decisions. The teams also need to understand why the architectural structures exist and the architecture's most important characteristics. This helps when the agile team works together with other teams when extending or adapting for a change in the architecture.

\subsection{Define components clearly}
As the systems evolve and the new components are introduced, the existing ones are seen to change, and the component interactions are altered. The teams need to define clear set of responsibilities and a set of required dependencies for the architecture and the design system to evolve coherently.

\subsection{Collaboration over Contracts}
The agile design should allow the teams to collaborate rather than maintain formal boundaries and agreements. Collaboration is the key to effective architecture work and is supported by the Agile manifesto.

\subsection{Focus design work on stakeholder concerns}
One of the aspects of collaboration is that the architectural design work aligns with the needs of the stakeholder and the teams working on it with others, this also works on systems most important aspects. This can be achieved by the inter working teams by using common design patterns, how the system will be deployed and other critical qualities such as security and availability of the system.

\section{Conclusion}
In future there are going to be more situations where multiple agile teams will have to work together cohesively towards a common goal to achieve a product. Therefore, it is very important for the teams to learn to collaborate effectively which can be achieved by clearly specifying the objectives for the team and segregating their work. Having a cross functional team in which the developers actively communicate between different teams thus actively avoiding any grey area and finally active collaboration between teams while deciding on the architecture and design so that all the stakeholders can give their input and design a comprehensive architecture with all the requirements and limitations in mind which can adopt future changes with ease. Thus, allowing for the smooth progress of the development without any sudden surprises like having a change in requirement that causes a disruptive changes to the product that has been built up to that point.

%
% The next two lines define the bibliography style to be used, and the bibliography file.
\bibliographystyle{ACM-Reference-Format}%alpha
\bibliography{reference}

\end{document}
